\input{usepackage.tex}
\begin{document}

\section{Auswertung}

Es wurde die Veränderung des optischen Drehwinkels $\alpha$ über die Zeit bei unterschiedlichen Temperaturen gemessen. Diese Werte wurden in Abbildung ref{at} dargestellt


\begin{figure}[h]
	\centering	
	\begin{minipage}{1\textwidth}
	\includegraphics[width=\columnwidth]{Bilder/Graph1.png}
	\end{minipage}
	
	
	\caption{Auftragung des optischen Drehwinkels gegen die Zeit bei 25°C, 31°C, 35.5°C, 41°C zur Bestimmung von \alpha_0, \alpha_\infty sowie von k. Die Auswertung erfolgte mit Igor Pro 6.37.}
	

	\label{at}
\end{figure}
%%%%%%%%%%%%%%%%%%%%%%%

\begin{equation}
k_T = k_\infty \cdot e^{\frac{E_a}{RT}
\end{equation}

\begin{equation}
ln(k_T) = ln(k_\infty) + \frac{E_a}{RT}
\end{equation}

\begin{equation}
ln(k_T) = \frac{E_a}{R} \cdot \frac{1}{T} + ln(k_\infty) 
\end{equation}




Anschießend wurden der Natürliche Logarithmus von $k_T$ gegen $\frac{1}{T}$ gemäß Gleichung REF aufgetragen. Durch eine lineare Regression konnten somit $E_a$ und $k_\infty$ bestimmt werden. Dies ist in Abbildung \ref{ln} dargestellt.


\begin{figure}[h]
	\centering	
	\begin{minipage}{1\textwidth}
	\includegraphics[width=\columnwidth]{Bilder/Graph2.png}
	\end{minipage}
	
	
	\caption{Auftragung von k_T über \frac{1}{T} zur Bestimmung von E_a und k_\infty mittels einer linearen regression. Die Auswertung erfolgte mit Igor Pro 6.37}
	
	
	\label{ln}
\end{figure}




\end{document}
