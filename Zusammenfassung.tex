\input{usepackage.tex}

 
\begin{document}
\section{Zusammenfassung}

Durch die Messung des optischen Drehwinkels der Hydrolyse von Saccharose über die Zeit mit $3-N$ HCl wurde die Reaktionskonstante bestimmt. Da dies bei unterschiedlichen Temperaturen geschah, konnte durch Auftragen des Natürliche Logarithmus von $k_T$ gegen $\frac{1}{T}$ aus der Steigung der angepassten Geraden $E_a$ und aus dem Ordinatenabschnitt $k_\infty$ errechnet werden. 

\begin{table}[H]
\centering
 
 
 \caption{Zusammenfassung der Ergebnisse der lineraren Regression der logarithmierten Arrhenius-Gleichung in Gegenüberstellung zur Literatur.}
\begin{tabular}{L{0.2\linewidth}L{0.2\linewidth}L{0.20\linewidth}R{0.2\linewidth}}
Konstante & Einheit & Messwert & Literatur \cite{saclit}\\
\hline \addlinespace[1ex] 
$ k_\infty$ & $ \frac{1}{s} $ & $ 2.77\cdot 10^{26}$  & $ 14 \cdot 10^{14}$\\
\addlinespace[1ex]
$ E_a $ & $ \frac{kJ}{mol} $ & $175 \pm 35$ & $108.4$\\


 \end{tabular}
 \label{tab2}
 \end{table}

Der gemessene Wert der Aktivierungsenergie liegt im zweifachen Fahlerintervall. Der Wert für $k_\infty$ liegt weit darüber, da geringe Abweichungen in den Temperaturen durch die Potenzierung schlussendlich zu großen Abweichungen führen.

\end{document}


