%\documentclass[a4paper, 12pt]{scrreprt}

\documentclass[a4paper, 12pt]{scrartcl}
%usepackage[german]{babel}
\usepackage{microtype}
%\usepackage{amsmath}
%usepackage{color}
\usepackage[utf8]{inputenc}
\usepackage[T1]{fontenc}
\usepackage{wrapfig}
\usepackage{lipsum}% Dummy-Text
\usepackage{multicol}
\usepackage{alltt}
%%%%%%%%%%%%bis hierhin alle nötigen userpackage
\usepackage{tabularx}
\usepackage[utf8]{inputenc}
\usepackage{amsmath}
\usepackage{amsfonts}
\usepackage{amssymb}
\usepackage{graphicx}
\usepackage{longtable}
\usepackage{booktabs}
\usepackage{tabularx}
\usepackage{siunitx}

%\usepackage{wrapfig}
\usepackage[ngerman]{babel}
\usepackage[left=25mm,top=25mm,right=25mm,bottom=25mm]{geometry}
%\usepackage{floatrow}
\setlength{\parindent}{0em}
\usepackage[font=footnotesize,labelfont=bf]{caption}
\numberwithin{figure}{section}
\numberwithin{table}{section}
\usepackage{subcaption}
\usepackage{float}
\usepackage{url}
\usepackage{fancyhdr}
\usepackage{array}
\usepackage{geometry}
%\usepackage[nottoc,numbib]{tocbibind}
\usepackage[pdfpagelabels=true]{hyperref}
\usepackage[font=footnotesize,labelfont=bf]{caption}
\usepackage[T1]{fontenc}
\usepackage {palatino}
%\usepackage[numbers,super]{natbib}
%\usepackage{textcomp}
\usepackage[version=4]{mhchem}

\usepackage{hyperref}

 
\begin{document}
\section{Zusammenfassung}

Durch die Messung des optischen Drehwinkels der Hydrolyse von Saccharose über die Zeit mit $3-N$ HCl wurde die Reaktionskonstante bestimmt. Da dies bei unterschiedlichen Temperaturen geschah, konnte durch Auftragen des Natürliche Logarithmus von $k_T$ gegen $\frac{1}{T}$ aus der Steigung der angepassten Geraden $E_a$ und aus dem Ordinatenabschnitt $k_\infty$ errechnet werden. 

\begin{table}[H]
\centering
 
 
 \caption{Zusammenfassung der Ergebnisse der lineraren Regression der logarithmierten Arrhenius-Gleichung in Gegenüberstellung zur Literatur.}
\begin{tabular}{L{0.2\linewidth}L{0.2\linewidth}L{0.20\linewidth}R{0.2\linewidth}}
Konstante & Einheit & Messwert & Literatur \cite{saclit}\\
\hline \addlinespace[1ex] 
$ k_\infty$ & $ \frac{1}{s} $ & $ 2.77\cdot 10^{26}$  & $ 14 \cdot 10^{14}$\\
\addlinespace[1ex]
$ E_a $ & $ \frac{kJ}{mol} $ & $175 \pm 35$ & $108.4$\\


 \end{tabular}
 \label{tab2}
 \end{table}

Der gemessene Wert der Aktivierungsenergie liegt im zweifachen Fahlerintervall. Der Wert für $k_\infty$ liegt weit darüber, da geringe Abweichungen in den Temperaturen durch die Potenzierung schlussendlich zu großen Abweichungen führen.

\end{document}


